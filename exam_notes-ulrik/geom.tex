\newcommand{\T}{\mathcal{T}}
\newcommand{\V}{\mathcal{V}}
\newcommand{\D}{\mathcal{V}}
\newcommand{\DG}{\mathcal{DG}}
\newcommand{\DT}{\mathcal{DT}}
\renewcommand{\S}{\mathcal{S}}
\newcommand{\Vor}{\mathrm{Vor}}

\begin{itemize}
\item Given a set $P \subset \mathcal{R}^2$ of points with
  measurements $f : P \rightarrow \mathcal{R}$, we may want to predict
  measurements at points not in $P$, based on $P$ and $f$.
\item Predicting based only on the nearest point gives an unnatural
  discrete approximation.
\item A better way is to determine a \emph{trianglulation} of $P$, and
  then predict the value of a given point $p$ by interpolating between
  the measurements at the three points that defines the triangle that
  $p$ lies within.
\item How do we determine the quality of a triangulation? Some
  triangulations look more natural than others.
\item It seems that triangles with small angles are bad. We rank
  triangulations by comparing their smallest angle.
\end{itemize}

\subsection{Triangulation of planar point sets}
\begin{itemize}
\item For a point set $P$, define a \emph{maximal planar subdivision}
  of $P$ as a set of edges $\mathcal{S}$ such that no edge connecting
  to vertices in $P$ can be added to $\mathcal{S}$ without destroying
  its planarity. I.e. any edge not in $\mathcal{S}$ intersects with
  some edge in $\mathcal{S}$.
\item A \emph{triangulation} $\mathcal{T}$ of $P$ is the maximal
  planar subdivision of $P$. Does such a division always consist of
  triangles? Yes, a bounded face is a polygon, and every polygon can
  be subdivided into triangles.
\item The number of triangles in a triangulation is $2n - 2 - k$,
  where $k$ is the number of points on the convex hull boundary, and
  $n$ is the number of points. The number of edges is $3n - 3 - k$.
\item We define the \emph{angle vector} $A(\T)$ of a triangulation
  $\T$ with $m$ triangles as a set of angles $\alpha_1, ...,
  \alpha_{3m}$. We say that $A(\T) > A(\T')$ when $\alpha_i =
  \alpha_i'$ for all $i < j$ and $\alpha_j > \alpha_j'$.
\item When $A(\T) \geq A(\T')$ for all triangulations $\T'$ of $P$, we
  say that $\T$ is \emph{angle-optimal}.
\item \emph{Thale's theorem} gives important inequalities (and
  equalities) about the smallest angle in triangles $p_ip_jp_k$ whose
  third point $p_k$ lies either inside, on, or outside the
  circumference of a circle with $p_i$ and $p_j$ on its circumference.
\item If edge $e = \overline{p_ip_j}$ is not an edge of the unbounded
  face, but incident to triangles $p_ip_jp_k$ and $p_ip_jp_l$ forming
  a convex quadrilateral, we can \emph{flip} $e$ by replacing it with
  $\overline{p_kp_k}$. This destroys the six angles
  $\alpha_1,...,\alpha_6$ and replaces them with
  $\alpha_1',...,\alpha_6'$.
\item We say than an edge $\overline{p_ip_j}$ is \emph{illegal} if
  $\min_{1 \leq i \leq 6} \alpha_i < \min_{1 \leq i \leq 6}
  \alpha_i'$.
\item Flipping an illegal edge in $\T$ results in a new triangulation
  $\T'$ with $A(\T') > A(\T)$.
\item We don't have to compute the angles to determine if an edge is
  illegal. Edge $\overline{p_ip_j}$ in triangles $p_ip_jp_k$,
  $p_ip_jp_l$ is illegal iff circle $C$ with $p_i,p_j,p_k$ on its
  circumference contains $p_l$.
\item A simple algorithm, \textsc{LegalTriangulation} transforms some
  triangulation $\T$ into a legal one by flipping illegal edges until
  all edges are legal. It terminates, as it is guaranteed to increase
  the angle value in each step. It is, however, inefficient.
\end{itemize}

\subsection{Delaunay triangulation}
\begin{itemize}
\item Given a set $P$ of $n$ $\emph{sites}$ in the plane, we can
  subdivide the plane into $n$ regions, one for each $p \in P$, such
  that for each region, every point in the region is closer to $p$
  than to any other site in $P$.
\item This is called the \emph{Voronoi diagram} for $P$, denoted
  $\Vor(P)$. Each cell is denoted $\V(p)$.
\item By connecting sites $p_ip_j$ by an edge whenever their Voronoi
  cells share a boundary, we obtain the \emph{Delaunay graph} for $P$,
  denoted $\DG(P)$.
\item $\DG(P)$ is \emph{planar}: No two edges cross.
\item $\DG(P)$ \emph{may} be a triangulation for $P$. In most cases it
  is, unless we have degenerate cases such as multiple sites lying on
  a common circle. A \emph{Delaunay triangulation} is obtained by
  adding missing edges to $\DG(P)$, since such cases always results in
  convex polygons.
\item From the properties of Voronoi diagrams, we have that all edges
  in a Delaunay triangulation are legal. Constructive proof: Since
  every Voronoi vertex is the center of a circle with exactly three
  sites on its circumference.
\item This also implies: Two points $p_i, p_j$ form an edge of iff
  there is a closed disc $C$ that has $p_i$ and $p_j$ on its boundary,
  and contains no other points in $P$.
\item A triangulation $\T$ is legal iff it is a $\DT(P)$.
\begin{proof}
  It follows trivially that any Delaunay triangulation is legal.
  \todo{Exercise other direction. See page 198.}
\end{proof}

\item All in all, we have that any Delaunay triangulation of $P$
  maximizes the minimum angle over all triangulations of $P$. (But
  they may not be angle optimal).
\end{itemize}

\subsection{Algorithm}

\begin{itemize}
\item There is an efficient way to compute the Delaunay triangulation
  directly, using an incremental algorithm.
\item We need to start with a valid triangulation. To do this, we add
  sites $p_{-1}, p_{-2}$, which together with the rightmost highest
  point $p_0$ forms a triangle ``large enough'' to contain all sites
  in $P$.
\begin{algorithm}
\caption{Delaunay Triangulation}
\begin{algorithmic}[1]
\Procedure{Delaunay-Triangulation}{$P$}

  \State Let $p_0$ be the highest rightmost point.
  \State Add points $p_{-1}, p_{-2}$, forming a triangle ``large
    enough'' to contain all sites in $P$.
  \State Let $\T$ be the triangulation with only $p_0p_{-1}p_{-2}$.
  \For{$r \leftarrow 1 \To n$}
    Find a triangle $t = p_ip_jp_k$ containing $p_r$.
    \If{$p_r$ lies on the interior of $t$}
      \State Add edges from $p_r$ to $p_i, p_j, p_k$, splitting $t$ into three triangles.
      \State \textsc{LegalizeEdge}$(p_r, \overline{p_ip_j}, \T)$
      \State \textsc{LegalizeEdge}$(p_r, \overline{p_jp_k}, \T)$
      \State \textsc{LegalizeEdge}$(p_r, \overline{p_kp_i}, \T)$
    \Else
      \Comment $p_r$ is on an edge of $t$, say $\overline{p_ip_j}$.
      \State Add edges to $p_k$ and to $p_l$, which is assumed to be in the other triangle for the edge $\overline{p_ip_j}$.
      \State \textsc{LegalizeEdge}$(p_r, \overline{p_ip_l}, \T)$
      \State \textsc{LegalizeEdge}$(p_r, \overline{p_lp_j}, \T)$
      \State \textsc{LegalizeEdge}$(p_r, \overline{p_jp_k}, \T)$
      \State \textsc{LegalizeEdge}$(p_r, \overline{p_kp_i}, \T)$
    \EndIf
  \EndFor

  \Return $\T$ with $p_{-1}, p_{-2}$ removed
\EndProcedure
\end{algorithmic}
\end{algorithm}

\item Adding new edges may make the triangulation illegal. We
  therefore flip edges until it is legal again.

  \begin{algorithm}
    \caption{Edge legalization}
    \begin{algorithmic}[1]
      \Procedure{Legalize-Edge}{$p_r, \overline{p_ip_j}, \T$}
        \If{$\overline{p_ip_j}$ is illegal}
          \State Remove $\overline{p_ip_j}$ from $\T$, and add $\overline{p_rp_k}$ instead.
          \State \textsc{LegalizeEdge}$(p_r, \overline{p_ip_k}, \T)$
          \State \textsc{LegalizeEdge}$(p_r, \overline{p_jp_k}, \T)$
        \EndIf
      \EndProcedure
    \end{algorithmic}
  \end{algorithm}

\item Any added edge is legal, and incident to the newly added point $p_r$.
  \begin{proof}
    Consider any point $p_r$ added to a triangulation $\T$, inside a
    triangle $p_ip_jp_k$. Before we added it, we could draw a circle
    $C$ with $p_i,p_j,p_k$ on its circumference, and no other points
    inside. Hence, we can ``contract'' $C$, creating a new circle $C'$
    with $p_i$ and $p_k$ on its circumference, and no other points
    inside. Hence, the edge $\overline{p_ip_k}$ is legal, and
    analogously for the two other edges.

    When flipping an edge, it will always end up being incident to
    $p_r$, and by the same logic as above, it will again become legal.
  \end{proof}

\item How do we find the triangle that the new point $p_r$ resides in?
  We maintain a directed acyclic graph $\D$ of triangles, which
  initially just contains $p_0p_{-1}p_{-2}$.

\item All leafs correspond to actual triangles in the current
  triangulation. When we split a triangle, we add pointers to each of
  the sub-triangles (max 3).

\item Expected running time is $O(n \log n)$. Expected storage is
  $O(n)$.
\end{itemize}

%%% Local Variables:
%%% mode: latex
%%% TeX-master: "notes"
%%% End: