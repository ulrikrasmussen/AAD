\begin{itemize}
\item For an NP-complete problem, we (supposedly) cannot find a
  polynomial algorithm for solving it.
\item Instead, we can try to brute force search the complete solution
  space, which is infeasible for all but the smallest problem
  instances. A more intelligent variant of brute-force is the
  \emph{branch-and-bound} method, which is a scheme for reducing the
  solution space.
\item Branch and bound gives no guarantees on the running time - it is
  still exponential in the worst case. However, we may get ``lucky''
  and be able to prune away very large subsets of the potential
  solution space.
\item Definition: For a given minimization problem with solution space
  $S$, assume we are given
  \begin{itemize}
  \item a method $\textsc{Branch}(S)$ which splits $S$ into some
    number of strictly smaller subsets,
  \item $f(s)$ which calculates the cost of some solution $s \in S$,
    and
  \item $lowerBound(S')$ which for some $S' \subseteq S$ computes a
    number $k$ such that $\forall s' \in S'. k \leq f(s')$.
  \end{itemize}
  \begin{algorithm}
    \caption{Branch and bound minimization scheme}
    \begin{algorithmic}[1]
      \Procedure{Branch-And-Bound-Min}{$S$}
        \State $Q = \{S\}$
        \State $best = \textsc{Nil}$
        \While{$Q \not= \emptyset$}
          \State Pick some $S_i \in Q$
          \If{$S_i = \{s\}$}
            \If{$best = \textsc{Nil} \lor f(s) < f(best)$}
              \State $best = s$
            \EndIf
          \Else
            \If{$lowerBound(S_i) < f(best)$}
              \State $Q = Q \cup \textsc{Branch}(S_i)$
            \EndIf
          \EndIf
        \EndWhile
        \Return $best$
      \EndProcedure
    \end{algorithmic}
  \end{algorithm}
\item Queue-popping strategy is usually \emph{best-first}, i.e. the
  subset $S_i$ which minimizes $lowerBound(S_i)$.
\item Example: TSP. Let $f(s)$ denote the length of a tour $s$, and
  let $\textsc{Branch}(S)$ split $S$ into two sets $S_1, S_2$ where
  some edge is included in all solutions in $S_1$ and exlcluded in all
  solutions in $S_2$.
\item $lowerBound(S')$ for TSP can be tricky. Several options:
  \begin{itemize}
  \item 1-tree. Pick some vertex $\mathbf{1} \in V$, and compute the
    MST over $V - \{ \mathbf{1} \}$. Then add $\mathbf{1}$ to the MST
    using the two minimal adjacent edges.

    This is a lower bound, as any hamiltonian tour with one edge
    removed will produce a spanning tree. Therefore, the MST plus the
    two minimal edges that will form a cycle is a lower bound for TSP.
  \item Linear program relaxation. For instance, pick the following
    mixed integer LP:
    \begin{align*}
      &\text{minimize} & \sum_{i \in V} \sum_{j \in V} d_{ij} x_{ij} \\
      &\text{subject to} & \sum_{i \in V} x_{ij} + \sum_{i \in V} x_{ji} ={}& 2 & \forall i \in V \\
      &                  & u_i - u_j + |V| \cdot x_{ij} \leq{}& |V| - 1 & \forall i,j \in V \\
      &                  & u_i \geq{}& 0 & \forall i \in V \\
      &                  & x_{ij} \in{}& \{0,1\} & \forall i,j \in V
    \end{align*}
    Simply relax by allowing $x_{ij}$ to take on any real value, then
    solve using simplex or similar
  \item Zones (with moats). Maximize radii of set of circles
    distributed over all vertices in the graph (formulate as
    LP). Using the fixed decision variables, form clusters of circles,
    and grow extra ``moats'' from these, such that they do not overlap
    with other circles or moats.

    In any tour, we must enter and exit each circle and
    moat. Therefore, a lower bound is two times the sum of all the
    radii and moat widths.
  \end{itemize}
\end{itemize}

%%% Local Variables:
%%% mode: latex
%%% TeX-master: "notes"
%%% End: