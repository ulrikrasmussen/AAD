% 5-1, 5-2, 5-3, 7.3, 9.2, 12.4, 35.4 (randomization) with a short
% overview of Chapter 7, without the details of the probabilistic
% analysis of Sections 9.2, 12.4 and of Chapter 7.

\subsection{Disposition}
\begin{itemize}
\item Explain basic idea, i.e. the use of probabilistic analysis to
  ensure a good expected running time.
\item Randomized quicksort
\item Randomly built binary search trees
\item Randomized 3-CNF approximation
%\item Randomized select
%\item In-place randomization
\end{itemize}

\subsection{Introduction}
\begin{itemize}
\item By making some assumptions about the distribution of our input,
  we can compute the average expected running time of an
  algorithm. This can sometimes provide a better upper bound on the
  \emph{average expected} running time.
\item We can rarely give any guarantees on the distribution of our
  input. Therefore, randomized algorithms use pseudo-random number
  generators to e.g. randomly permute a list of inputs, or to draw
  random samples from some distribution.
\item Hence, we can use probabilistic analysis with our new guarantees
  to derive an upper bound on the average expected running time,
  \emph{independent of the input}.
\end{itemize}

\subsection{Randomized quicksort}
\begin{itemize}
\item If we make no assumptions on the input to the deterministic
  quicksort algorithm, the average running time will be $O(n^2)$ - we
  could be so unlucky that the worst case running time occurs every
  time.
\item Instead, we can use \emph{randomized quicksort}, where the pivot
  element is chosen randomly with equal probabilities among the
  elements of the current partition.
\item Since each pair of elements can \emph{at most} be compared once
  (since we remove the pivot element from the two partitions), the
  number of total comparisons will be
$$
 X = \sum_{i=1}^{n-1} \sum_{j=i+1}^n X_{ij},
$$
where $X_{ij}$ is a stochastic variable with $x_{ij} = 1$ if elements
$z_i,z_j$ were compared, and $x_{ij} = 0$ otherwise.

\item Using linearity of expectations, the expected number of
  comparisons $E[X]$ is therefore
$$
 E[X] = \sum_{i=1}^{n-1} \sum_{j=i+1}^n E[X] = \sum_{i=1}^{n-1} \sum_{j=i+1}^n Pr \{\text{$z_i$ is compared to $z_j$}\}.
$$

\item We observe that for some partition $Z_{ij}$ of elements from
  $z_i$ to $z_j$, $z_i$ will only be compared to $z_j$ if $z_i$ was
  chosen as pivot, or if $z_j$ was chosen as pivot. Since there are $j
  - i + 1$ elements in $Z_{ij}$, we therefore have
$$
  E[X_{ij}] = Pr\{\text{$z_i$ pivot}\} + Pr\{\text{$z_j$ pivot}\} = 2 \cdot \frac{1}{j-i+1}
$$

\item Hence,
\begin{align*}
 E[X] ={}& \sum_{i=1}^{n-1} \sum_{j=i+1}^n \frac{2}{j-i+1} \\
      ={}& \sum_{i=1}^{n-1} \sum_{k=1}^{n-i} \frac{2}{k+1} \\
      <{}& \sum_{i=1}^{n-1} \sum_{k=1}^n \frac{2}{k} \\
      ={}& \sum_{i=1}^{n-1} O(\lg n) \\
      ={}& O(n \lg n).
\end{align*}
\end{itemize}


\subsection{Randomized 3-CNF approximation}

\begin{itemize}
\item We can develop a randomzed $8/7$-approximation algorithm to
  approximate MAX-3-CNF, to corresponding optimization problem to the
  NP-complete 3-CNF problem.

\item Assuming that each clause in the 3-CNF instance does not contain
  a variable and its negation, a randomized algorithm that
  independently and uniformly assigns $1$ or $0$ to each variable with
  probability $1/2$ for each, is a randomized $8/7$ approximation
  algorithm.
  \begin{proof}
    For the stochastic variable $Y_i$, we have $Y_i = 1$ if clause $i$
    is satisfied, and $Y_i = 0$ otherwise.  For a given clause $(x
    \lor y \lor z)$, only one assignment will make the clause false,
    namely the assignment that makes all terms $x,y,z$ become
    false. Since there are $2^3 = 8$ possible outcomes (assuming that
    the clause contains only distinct variables), this gives us a
    probability of $7/8$ for a given clause to become true. Hence
    $E[Y_i] = 7/8$.

    We are interested in computing
    $$
    C = E[Y] = \sum_{i=1}^m E[Y_i] = \sum_{i=1}^m 7/8 = m \frac{7}{8}.
    $$

    The upper bound for a solution is $m$, and hence the approximation
    ratio is bounded by $m / C = 8/7$.
  \end{proof}
\end{itemize}

\subsection{Randomly built binary search trees}

%%% Local Variables:
%%% mode: latex
%%% TeX-master: "notes"
%%% End: