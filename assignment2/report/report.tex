\documentclass[a4paper, 10pt, oneside, article]{memoir}
\chapterstyle{culver}
\checkandfixthelayout

\usepackage{lipsum}

% Palatino font
\usepackage{palatino}

% Font and input encoding
\usepackage[T1]{fontenc}
\usepackage[utf8x]{inputenc}

% Babel (language)
\usepackage[english]{babel}

% Support for blackboard bold symbols.
\usepackage{bbm}

% AMS-Math packages
\usepackage{amsmath}
\usepackage{amssymb}
\usepackage{amsthm}

% For including bitmap graphics
\usepackage{graphicx}

\usepackage{multirow}

\usepackage{todonotes}

% Remove chapters from figure counters
%\counterwithout{figure}{chapter}
% Add sections instead
%\counterwithin{figure}{section}
%\counterwithout{section}{chapter}

\usepackage{tikz}
\usetikzlibrary{shapes,arrows}

\title{Advanced Algorithms and Data Structures\\ Branch and Bound}
\author{Jens Raaby, Mads Hartmann \& Ulrik Rasmussen}

\begin{document}

\maketitle

\section*{Question 1}

% TODO: We probably need to make this more mathy
%       but the intuition should be sound 

Given a graph $G$ a 1-tree bound is defined as follows 

\begin{itemize}
  \item You select a vertex $v_1$ from $G$
  \item Remove all edges incident on $v_1$ from $G$ producing $G'$
  \item Find the minimum spanning tree $T$ of $G'$
  \item Add the two smallest edges $e_1$ and $e_2$ incident on $v_1$ to $T$ producing a 1-tree $T_1$
\end{itemize}

We're asked to show that the 1-tree is a lower bound for TSP; that is,
that the weight of the Hamiltonian tour of G with the lowest weight
is equal to or greater than weight  of $T_1$.

If we assume we have a Hamiltonian tour of $G$ that has has a lower
weight than the $T_1$ then it would require that there exists a
spanning tree of $G'$ that  has a lower wight than $T$; this is
impossible since $T$ is a minimum spanning tree. Here's why: A Hamiltonian
tour of $G$ will also need to be a Hamiltonian path of $G'$. In the best case 
the Hamiltonian path $H$ and the minimum spanning tree $T$ will be the same. Since we're
adding the two edges $e_1, e_2$ with the lowest weight then the only way for
$TSP$ to have a lower weight than $T_1$ would be if the weight of $H$ was smaller 
than $T$; a contradiction.





\section*{Question 2}
In this part we consider representing each monument as a circle, so that when all circles are maximum size and do not overlap we have a way of travelling to each monument by entering and leaving every circle.

To form a linear program to maximise the radii of these circles, it is a simple matter of representing each monument with a radius variable and adding constraints such that any two circles to not overlap.

\begin{align*}
  \text{maximize} && \sum_{v\in V} radius_{v} \\
  \text{subject to} && radius_{u} + radius_{v} \leq{}& distance(u,v) & \text{for each $(u,v) \in E$}, \\
                    && r_{v} \geq{}& 0 & \text{for each $v \in V$}.
\end{align*}

\section*{Question 3}
How do you compute a lower bound from these circles?

Considering any single monument, its distance from any other monument is at least the radius of that monument's circle. Since visiting all the monuments requires both entering and leaving each of the circles, a lower bound for a TSP problem could be expressed as $2 \sum_{v \in V} radius_{v}$, in other words, double the sum of all the radii.
This will be a fairly good lower bound if all the monuments are similar distances apart. However, if many of the monuments are located in small areas, then the radii will also be quite small. There may therefore be a much larger distance between these groups of monuments, meaning the actual TSP solution has a much higher cost.
% See http://www.tsp.gatech.edu/methods/opt/zone.htm


\section*{Question 4}

We can use the mixed integer linear formulation by Miller, Tucker and
Zemlin to obtain a linear program of polynomial size with regards to
the number of nodes in the TSP problem. In the following, $d_{ij}$
denotes the distance from node $i$ to node $j$, and $x_{ij}$ is an
integer variable in the domain $\{0, 1\}$, where $x_{ij} = 1$ iff the
edge from node $i$ to $j$ is included in the solution. In addition,
there are $|V|-1$ auxiliary variables $u_i$ in the domain of the real
numbers, which are used to eliminate subtours:

\begin{align*}
  &\text{minimize} &&& \sum_{i\in V} \sum_{j \in V} d_{ij} x_{ij} \\
  &\text{subject to} &&& \sum_{j \in V} x_{ij} ={}& 1 && \forall i \in V \\
  &                  &&& \sum_{i \in V} x_{ij} ={}& 1 && \forall j \in V \\
  &                  &&& u_i - u_j + |V| \cdot x_{ij} \leq{}& |V| - 1 && \forall i,j \in V - \{1\}  \\
  &                  &&& u_i \geq{}& 0 && \forall i \in V - \{1\} \\
  &                  &&& x_{ij} \in{}& \{0, 1\} && \forall i,j \in V
\end{align*}

The first two constraints ensures that each node has exactly one
incoming and one outgoing edge that is included in the solution. The
third constraint ensures that there is only a single tour in the
solution, namely the tour beginning and ending at node $1$.


\section*{Question 5}

We can relax the above mixed integer linear program by allowing the
edge variables $x_{ij}$ to take on real values. This yields the
following linear program, which does not include integer variables,
and hence can be solved efficiently:
\begin{align*}
  &\text{minimize} &&& \sum_{i\in V} \sum_{j \in V} d_{ij} x_{ij} \\
  &\text{subject to} &&& \sum_{j \in V} x_{ij} ={}& 1 && \forall i \in V \\
  &                  &&& \sum_{i \in V} x_{ij} ={}& 1 && \forall j \in V \\
  &                  &&& u_i - u_j + |V| \cdot x_{ij} \leq{}& |V| - 1 && \forall i,j \in V - \{1\}  \\
  &                  &&& u_i \geq{}& 0 && \forall i \in V - \{1\} \\
  &                  &&& x_{ij} \leq{}& 1 && \forall i,j \in V \\
  &                  &&& x_{ij} \geq{}& 0 && \forall i,j \in V
\end{align*}

The solution space for the original integer linear program is a subset
of the solution space for the relaxed version. Hence an optimal
solution to the relaxed program will be a lower bound for the integer
solution.

\section*{Question 6}

\begin{tabular}{r | c c c }

 & Instance 1 & Instance 2 & Instance 3 \\
\hline

Brute-force & 8.649, 233902 & 19.030, 1659236 & \emph{inf} \\

1-Tree      & 8.649, 233902 & 19.030, 1659236 & \emph{inf} \\

2           & r3c2          & r3c3            & r3c4\\

Relax       & r4c2          & r4c3            & r4c4\\

\end{tabular}


\section*{Question 7}

\textbf{TODO}

\bibliographystyle{abbrv}
\bibliography{bibliography}
\end{document}
